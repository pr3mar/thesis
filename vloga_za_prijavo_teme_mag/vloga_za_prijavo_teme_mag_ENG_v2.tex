\documentclass[a4paper, 12pt]{article}
\usepackage[slovene]{babel}
\usepackage{lmodern}
\usepackage[T1]{fontenc}
\usepackage[utf8]{inputenc}
\usepackage{url}
\usepackage{hyperref}

\topmargin=0cm
\topskip=0cm
\textheight=25cm
\headheight=0cm
\headsep=0cm
\oddsidemargin=0cm
\evensidemargin=0cm
\textwidth=16cm
\parindent=0cm
\parskip=12pt

\renewcommand{\baselinestretch}{1.2}

\begin{document}

%%%%%%%%%%%%%%%%%%%%%%%%%% Filled out by the candidate! %%%%%%%%%%%%%%%%%%%%%%%%%%
\newcommand{\ImeKandidata}{Marko} % Name
\newcommand{\PriimekKandidata}{Prelevikj} % Surname
\newcommand{\VpisnaStevilka}{63130345} % Enrollment number
\newcommand{\StudyProgramme}{Computer and information science, MAG} % Study programme
\newcommand{\NaslovBivalisca}{Trg Prekomorskih Brigad 11, 1000 Ljubljana, Slovenia} % the candidate address
\newcommand{\SLONaslov}{Vodenje projektov na podlagi analize podatkov} % Slovenian title
\newcommand{\ENGNaslov}{Data Driven Project Management} % English title
%%%%%%%%%%%%%%%%%%%%%%%%%% End of filling in  %%%%%%%%%%%%%%%%%%%%%%%%%%


\newcommand{\Kandidat}{\ImeKandidata~\PriimekKandidata}
\noindent
\Kandidat\\
\NaslovBivalisca \\
Study programme: \StudyProgramme \\
Enrollment number: \VpisnaStevilka
\bigskip

{\bf Committee for Student Affairs}\\
Univerza v Ljubljani, Fakulteta za računalništvo in informatiko\\
Večna pot 113, 1000 Ljubljana\\

{\Large\bf
{\centering
    The master’s thesis topic proposal \\%[2mm]
\large Candidate: \Kandidat \\[10mm]}}


I, \Kandidat, a student of the 2nd cycle study programme at the Faculty of computer and information science, am submitting a thesis topic proposal to be considered by the Committee for Student Affairs with the following title:

%\hfill\begin{minipage}{\dimexpr\textwidth-2cm}
Slovenian: {\bf \SLONaslov}\\
English: {\bf \ENGNaslov}
%\end{minipage}

This topic was already approved last year: {\bf \textit{NO}}

I declare that the mentor listed below have approved the submission of the thesis topic proposal described in the remainder of this document.

I would like to write the thesis in English with the following reason: I am a foreigner and more experienced with writing in English.

I propose the following mentor:

%%%%%%%%%%%%%%%%%%%%%%%%%% Filled in by the candidate! %%%%%%%%%%%%%%%%%%%%%%%%%%
\hfill\begin{minipage}{\dimexpr\textwidth-2cm}
Jure Demšar, doc. dr. \\
University of Ljubljana, \\
Faculty of Computer and Information Science\\
\href{mailto:jure.demsar@fri.uni-lj.si}{jure.demsar@fri.uni-lj.si}
\end{minipage}
%%%%%%%%%%%%%%%%%%%%%%%%%% End of filling in %%%%%%%%%%%%%%%%%%%%%%%%%%

\bigskip

\hfill Ljubljana, \today.


\clearpage
\section*{Proposal of the masters thesis topic}

\section{The narrow field of the thesis topic}

English: project management, task workflow analysis, data analysis

Slovene: vodjenje projektov, analiza poteka dela, podatkovna analiza

\section{Keywords}

English: agile project management, project management information system, quantitative data analysis, project success, performance metrics

Slovene: agilno vodjenje projektov, informacijski sistemi za podporo vodjenje projektov, kvantitativna podatkovna analiza, uspeh projekta, meritve uspe"snosti

\section{Detailed thesis proposal}

% Navodilo (pobrišite v končnem izdelku):
% V nadaljevanju opredelite izhodišča magistrskega dela in utemeljite znanstveno ali strokovno relevantnost predlagane teme.

\textbf{Past approvements of the proposed thesis topic:}\\
The proposed thesis has not been submitted nor approved in previous years.

\subsection{Introduction and problem formulation}

%Guideline (delete from the final version):
%Explain the problem which you plan to tackle and motivate your work. In motivation, relate to the literature and unsolved problems that your thesis will address. Position your work within the narrow field of research. The text should be approximately 800 characters long, including spaces.

Project managers (PMs) have the role of leading the team towards the achieving of the project objectives~\cite{institute2017guide}. Project Management Information Systems (PMIS) are used by PMs to assist their decision making for planning, organizing and controlling projects~\cite{CANIELS2012162}. PMIS keep the state of the organization's projects and are visualizing it with Burn-Down Charts, Gantt Charts, or other basic visualisations which provide merely a high-level overview of metrics such as work completed, story points completed, deliverable status~\cite{institute2017guide}, etc. As such, PMIS do not provide enough useful information to PMs~\cite{CANIELS2012162}. 

The value of PMIS drops due to their elementary reporting abilities. This leads to PMs having to do double the work: keeping the PMIS up to date and analysing the data separately to support the decision making.

%Which leads to PMIS being an unused source of knowledge.
%for a greater purpose: to assess the members and identify the bottlenecks within the project's workflows.
%~\footnote{In agile environments the Burn-Down Charts seem to be more useful because they visualize the percent of the project's deliverables, however, in traditional, waterfall projects Gantt Charts are used because they visualize the state at which each step of the project is at.}

%However, the data contains much more unused information than it has been initially anticipated. 

%Project Management Information Systems (PMIS)~\cite{institute2017guide} provide structure to the projects which support project managers' (PM) task of monitoring and optimizing the organizations processes. Whereas the PMIS' role is to keep the data well-structured and easily manageable, the PMIS usually provide only a low level of reporting capability. The basic reports and the visualizations provided by the PMIS do not provide much semantic insight from the underlying data, which is often of critical significance. 

\subsection{Related work}

%Guideline (delete from the final version):
%Prepare an overview of the related work that is directly related to your problem. Describe the main highlights of each related work in a few sentences. Point out pros and drawbacks. Reference the works listed in the References section. The related work should be focused and approximately half of A4 page long.


The use of PMIS is a common and widespread practice across enterprises. It has been shown that they have a direct impact on the project success~\cite{RAYMOND2008213}, as they provide a structured overview of the project state and support the decision making process of PMs. Contemporary PMIS provide organization-wide transparency and their usage is not limited to only PMs, but it is instead widespread over the majority of the organization's members. This is especially important within agile environments where every member is tracking their own progress. Agile project management has been gaining momentum ever since the appearance of The Agile Manifesto~\cite{alliance2001agile} because it has been shown to work in practice~\cite{SERRADOR20151040}, i.e. the success rate of projects with agile methodology is greater than traditional projects. 

Our aim is to improve the general usability of contemporary PMIS which are used within projects which have applied agile project methodology. An example of such PMIS is Atlassian's \href{https://www.atlassian.com/software/jira}{JIRA}, where we have the unique advantage of observing the users' habits of using the PMIS by examining the history of their usage: how often they interact with the PMIS, what do they edit, which workflow element are their tasks most often in, etc.

By examining the history of the user's interactions with the PMIS we expect that we will be able to perform an analysis of the causality of the users actions and try to identify the possible bottlenecks which are preventing the projects from moving forward. A concern has been brought up by~\cite{tavares2019risk} that SCRUM projects~\cite{sutherland2013scrum} are susceptible to potential risk due to the fact that they do not have a specific process of risk management application, even though it is vital for the success of the project. A similar research has been performed within~\cite{Toole2006APM}, where they were able to classify how each of the performed actions influenced on the project: whether it had a positive or a negative effect.

%
%as they will be able to steer the project in the right direction, i.e. towards success. 
%
%It is also worth mentioning that having the causality is also helping PMs 
%
%The order of the following articles is very important, it explains the flow of the story:
%
%\begin{enumerate}
%	\item \cite{CANIELS2012162}: why we need PMIS  (Project Management Information Systems), how do we benefit from it. We can use it to motivate the usage of PMIS
%
%	\item \cite{RAYMOND2008213}: An old overview of PMIS. It is often cited. This is related just so we can identify what kinds of PMIS there are and what we are going to work with.
%
%	\item \cite{SERRADOR20151040}: a quantitative analysis of whether or not agile works. Why we are sticking with agile and how we can exploit it the most.
%
%	\item \cite{Toole2006APM}: has a good point about setting priorities and the causality of the actions. What actions we take and how they affect the final outcome
%
%	\item \cite{tavares2019risk}: This has a good point as well, how is risk handled in SCRUM projects. It fits well in the story because we'd like to minimize the risk and maximize the output/throughput of the project
%
%\end{enumerate}
%
%The problem that I have is how do I state that this is relevant to what I want to work on.
%
%The task at hand is not actually deeply connected to project management, but it is rather making the project management easier:
%we want to perform analysis of the usage data of a particular PMIS (JIRA) and identify the potential risks (e.g. outliers) based on the analysis.
%
%For example, we would be able to identify what are the bottlenecks of the workflow: is it a person? Is it the workflow itself?
%Can we make a workaround? Can we make a recommendation to the PM/developer on how to proceed? Who to assign the task to next?



\subsection{Expected contributions}

%Guideline (delete from the final version):
%Describe the expected contributions of your masters thesis in the field of computer science. The contributions can be either scientific or technical. Describe the novelties of your contribution in relation to the related work and state-of-the-art (scientific or technical-wise). The description should be approximately 500 characters long, including spaces.

Our first contribution is the validation of our hypothesis 

\textbf{H1}: Applying modern data analysis techniques to data generated by a contemporary PMIS can help us extract insights from which the underlying enterprise will benefit. 

Under the assumption that our hypothesis is confirmed, our second contribution would be a prototype of a Project Management Support Tool (PMST) which offers benefits in optimizing the workflows of the projects and identifying the outliers of the parties involved.

\subsection{Methodology}

%Guideline (delete from the final version):
%Briefly describe the methodology that you intend to apply in your work. This section should describe the methods you intend to use (e.g., the framework used for development, theoretical frameworks), methods that will be applied for analysis and evaluation of your approach, and comparison with the most related works. The description should be approximately 500 characters long, including spaces.

In order to achieve our expected contributions we need to analyse the data from a contemporary PMIS, in our case that will be Atlassian's \href{https://www.atlassian.com/software/jira}{JIRA}. Our plan is consisted of firstly performing ECTL processing on the data, in order to prepare the data for analysis. And then applying a variety of predominantly, but not necessarily limited to, statistical methods on the data to validate \textbf{H1}. In the case of confirmation of our hypothesis, we would continue to generalize our analysis in a form of prototype of a PMST.

%This is pretty vague as well, I am having difficulties identifying the things I can do with the data, let alone identifying the methodology of how I'm going to do that. 

%The goal is to achieve some degree of \textbf{life-cycle analysis of the tasks}.
%
%For example, there are a couple of ideas of what we can do to achieve it:
%
%\begin{itemize}
%	\item analyse the time spent in a certain state of the workflow
%	
%	\item identify the people that are halting the process and recommend replacements
%
%	\item create distributions of different meta-data of the tasks: \#comments, \#changes, \# labels, \dots. All in order to find the correlation of some of the attributes (meta data fields): e.g. high priority cards have a high rate of state change in a very narrow time span.
%	
%	\item \dots
%\end{itemize}
%
%All of these are ideas which seem like they do not require any complicated analysis in order to achieve.
%
%The approach should be iterative: first do some ``shallow`` analysis in order to find something interesting, and then once we verify we bring out the ``big guns``.

\subsection{References}
\label{literature}


%Guideline (delete from the final version):
%List all the references that you cite in the proposal. Use the scientific standard of citing, e.g. \cite{Zivkovic2004}. The list should contain at least a few works published in the recent years. Preferably, the references should include publications from conferences, journals or other well-accepted sources in your field.

\renewcommand\refname{}
\vspace{-50px}
\bibliographystyle{elsarticle-num}
\bibliography{./bibliografija/bibliography}


%\bigskip
%
%Ljubljana, \today .

\end{document}
