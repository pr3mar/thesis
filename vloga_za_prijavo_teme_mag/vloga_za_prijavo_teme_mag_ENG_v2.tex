\documentclass[a4paper, 12pt]{article}
\usepackage[slovene]{babel}
\usepackage{lmodern}
\usepackage[T1]{fontenc}
\usepackage[utf8]{inputenc}
\usepackage{url}
\usepackage{hyperref}

\topmargin=0cm
\topskip=0cm
\textheight=25cm
\headheight=0cm
\headsep=0cm
\oddsidemargin=0cm
\evensidemargin=0cm
\textwidth=16cm
\parindent=0cm
\parskip=12pt

\renewcommand{\baselinestretch}{1.2}

\begin{document}

%%%%%%%%%%%%%%%%%%%%%%%%%% Filled out by the candidate! %%%%%%%%%%%%%%%%%%%%%%%%%%
\newcommand{\ImeKandidata}{Marko} % Name
\newcommand{\PriimekKandidata}{Prelevikj} % Surname
\newcommand{\VpisnaStevilka}{63130345} % Enrollment number
\newcommand{\StudyProgramme}{Computer and information science, MAG} % Study programme
\newcommand{\NaslovBivalisca}{Trg Prekomorskih Brigad 11, 1000 Ljubljana, Slovenia} % the candidate address
\newcommand{\SLONaslov}{Vodenje projektov na podlagi analize podatkov} % Slovenian title
\newcommand{\ENGNaslov}{Data Driven Project Management} % English title
%%%%%%%%%%%%%%%%%%%%%%%%%% End of filling in  %%%%%%%%%%%%%%%%%%%%%%%%%%


\newcommand{\Kandidat}{\ImeKandidata~\PriimekKandidata}
\noindent
\Kandidat\\
\NaslovBivalisca \\
Study programme: \StudyProgramme \\
Enrollment number: \VpisnaStevilka
\bigskip

{\bf Committee for Student Affairs}\\
Univerza v Ljubljani, Fakulteta za računalništvo in informatiko\\
Večna pot 113, 1000 Ljubljana\\

{\Large\bf
{\centering
    The master’s thesis topic proposal \\%[2mm]
\large Candidate: \Kandidat \\[10mm]}}


I, \Kandidat, a student of the 2nd cycle study programme at the Faculty of computer and information science, am submitting a thesis topic proposal to be considered by the Committee for Student Affairs with the following title:

%\hfill\begin{minipage}{\dimexpr\textwidth-2cm}
Slovenian: {\bf \SLONaslov}\\
English: {\bf \ENGNaslov}
%\end{minipage}

This topic was already approved last year: {\bf \textit{NO}}

I declare that the mentor listed below have approved the submission of the thesis topic proposal described in the remainder of this document.

I would like to write the thesis in English with the following reason: I am a foreigner and more experienced with writing in English.

I propose the following mentor:

%%%%%%%%%%%%%%%%%%%%%%%%%% Filled in by the candidate! %%%%%%%%%%%%%%%%%%%%%%%%%%
\hfill\begin{minipage}{\dimexpr\textwidth-2cm}
Jure Demšar, doc. dr. \\
University of Ljubljana, \\
Faculty of Computer and Information Science\\
\href{mailto:jure.demsar@fri.uni-lj.si}{jure.demsar@fri.uni-lj.si}
\end{minipage}
%%%%%%%%%%%%%%%%%%%%%%%%%% End of filling in %%%%%%%%%%%%%%%%%%%%%%%%%%

\bigskip

\hfill Ljubljana, \today.


\clearpage
\section*{Proposal of the masters thesis topic}

\section{The narrow field of the thesis topic}

English: agile project management, task workflow analysis


\section{Keywords}

English: agile project management, project management support tool, quantitative analysis
% project management information system,


\section{Detailed thesis proposal}

% Navodilo (pobrišite v končnem izdelku):
% V nadaljevanju opredelite izhodišča magistrskega dela in utemeljite znanstveno ali strokovno relevantnost predlagane teme.

\textbf{Past approvements of the proposed thesis topic:}\\
The proposed thesis has not been submitted nor approved in previous years.

\subsection{Introduction and problem formulation}

%Guideline (delete from the final version):
%Explain the problem which you plan to tackle and motivate your work. In motivation, relate to the literature and unsolved problems that your thesis will address. Position your work within the narrow field of research. The text should be approximately 800 characters long, including spaces.

Project Management Information Systems (\textit{PMIS}) provide different forms of structure to the projects which support \textit{project managers}' (\textit{PM}) task of monitoring and optimizing the organizations processes. Whereas the \textit{PMIS}'s role is to keep the data well-structured and easily manageable, the \textit{PMIS} usually provide only a low level of reporting capability. The basic reports and the visualizations provided by the \textit{PMIS} do not provide much semantic insight from the underlying data, which is often of critical significance. 

\subsection{Related work}

%Guideline (delete from the final version):
%Prepare an overview of the related work that is directly related to your problem. Describe the main highlights of each related work in a few sentences. Point out pros and drawbacks. Reference the works listed in the References section. The related work should be focused and approximately half of A4 page long.

~\cite{RAYMOND2008213}

\subsection{Expected contributions}

%Guideline (delete from the final version):
%Describe the expected contributions of your masters thesis in the field of computer science. The contributions can be either scientific or technical. Describe the novelties of your contribution in relation to the related work and state-of-the-art (scientific or technical-wise). The description should be approximately 500 characters long, including spaces.

The final outcome of the research is expected to be a \textit{Project Management Support Tool} (\textit{PMST}) which offers targeted benefits to both project managers and developers which are not yet offered as such.\textit{PMST} helps project managers in optimizing the enterprises' projects' workflows, identifying outliers within the organization, and identifying implicit shift of priorities of tasks. On the other hand, the \textit{PMST} offers developers a tool for maximizing the productivity within their existing workflow.

\subsection{Methodology}

%Guideline (delete from the final version):
%Briefly describe the methodology that you intend to apply in your work. This section should describe the methods you intend to use (e.g., the framework used for development, theoretical frameworks), methods that will be applied for analysis and evaluation of your approach, and comparison with the most related works. The description should be approximately 500 characters long, including spaces.


\subsection{References}
\label{literature}


%Guideline (delete from the final version):
%List all the references that you cite in the proposal. Use the scientific standard of citing, e.g. \cite{Zivkovic2004}. The list should contain at least a few works published in the recent years. Preferably, the references should include publications from conferences, journals or other well-accepted sources in your field.

\renewcommand\refname{}
\vspace{-50px}
\bibliographystyle{elsarticle-num}
\bibliography{./bibliografija/bibliography}


%\bigskip
%
%Ljubljana, \today .

\end{document}
