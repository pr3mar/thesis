\documentclass[a4paper, 12pt]{article}
\usepackage[slovene]{babel}
\usepackage{lmodern}
\usepackage[T1]{fontenc}
\usepackage[utf8]{inputenc}
\usepackage{url}

\topmargin=0cm
\topskip=0cm
\textheight=25cm
\headheight=0cm
\headsep=0cm
\oddsidemargin=0cm
\evensidemargin=0cm
\textwidth=16cm
\parindent=0cm
\parskip=12pt

\renewcommand{\baselinestretch}{1.2}

\begin{document}

%%%%%%%%%%%%%%%%%%%%%%%%%% Izpolni kandidat! %%%%%%%%%%%%%%%%%%%%%%%%%%
\newcommand{\ImeKandidata}{Ime} % Ime
\newcommand{\PriimekKandidata}{Priimek} % Priimek
\newcommand{\VpisnaStevilka}{60606060 } % vpisna številka
\newcommand{\StudijskiProgram}{Računalništvo in informatika, MAG} % Študijski program/smer
\newcommand{\NaslovBivalisca}{ Cesta 1, 1234 Mesto, Slovenija } % kaniddatov naslov
\newcommand{\SLONaslov}{Naslov dela v slovenščini} % naslov dela v slovenščini
\newcommand{\ENGNaslov}{Title of the thesis in English} % naslov dela v angleščini
%%%%%%%%%%%%%%%%%%%%%%%%%% Konec izpolnjevanja %%%%%%%%%%%%%%%%%%%%%%%%%%


\newcommand{\Kandidat}{\ImeKandidata~\PriimekKandidata}
\noindent
\Kandidat\\
\NaslovBivalisca \\
Študijski program: \StudijskiProgram \\
Vpisna številka: \VpisnaStevilka
\bigskip

{\bf Komisija za študijske zadeve}\\
Univerza v Ljubljani, Fakulteta za računalništvo in informatiko\\
Večna pot 113, 1000 Ljubljana\\

{\Large\bf
{\centering
    Vloga za prijavo teme magistrskega dela \\%[2mm]
\large Kandidat: \Kandidat \\[10mm]}}


\Kandidat, študent/-ka magistrskega programa na Fakulteti za računalništvo in informatiko, zaprošam Komisijo za študijske zadeve, da odobri predloženo temo magistrskega dela z naslovom:

%\hfill\begin{minipage}{\dimexpr\textwidth-2cm}
Slovenski: {\bf \SLONaslov}\\
Angleški: {\bf \ENGNaslov}
%\end{minipage}

Gre za ponovno vloženo temo, ki je bila lani že potrjena: {\bf \textit{DA} , \textit{NE} (izberite en odgovor) }

Izjavljam, da so spodaj navedeni mentorji predlog teme pregledali in odobrili ter da se z oddajo predloga strinjajo.

Magistrsko delo nameravam pisati v slovenščini. % In case you would like to write the thesis in English, comment this line out, and use the following template to explain your request:
%Komisijo zaprošam, da odobri pisanje magistrskega dela v angleškem jeziku z obrazložitvijo ... .

Za mentorja/mentorico predlagam:

%%%%%%%%%%%%%%%%%%%%%%%%%% Izpolni kandidat! %%%%%%%%%%%%%%%%%%%%%%%%%%
\hfill\begin{minipage}{\dimexpr\textwidth-2cm}
Ime in priimek, naziv: \\
Ustanova: \\
Elektronski naslov:
\end{minipage}

Za somentorja/somentorico predlagam:

\hfill\begin{minipage}{\dimexpr\textwidth-2cm}
Ime in priimek, naziv: \\
Ustanova: \\
Elektronski naslov: \\
\end{minipage}



%%%%%%%%%%%%%%%%%%%%%%%%%% Konec izpolnjevanja %%%%%%%%%%%%%%%%%%%%%%%%%%

\bigskip


\hfill V Ljubljani, dne \today.
%V Ljubljani, dne …………………………
%
%Podpis mentorja: \hspace{180px} Podpis kandidata/kandidatke:




\clearpage
\section*{PREDLOG TEME MAGISTRSKEGA DELA}

\section{Področje magistrskega dela}

slovensko: npr. računalništvo in informatika, računalniška arhitektura\\
angleško: e.g. computer science, computer architecture


\section{Ključne besede}

slovensko:    \\
angleško:


\section{Opis teme magistrskega dela}

% Navodilo (pobrišite v končnem izdelku):
% V nadaljevanju opredelite izhodišča magistrskega dela in utemeljite znanstveno ali strokovno relevantnost predlagane teme.

\textbf{Pretekle potrditve predložene teme:}\\
Predložena tema ni bila oddana in potrjena v preteklih letih.
% Tu ne gre za to, da bi morali pogledati vse pretekle teme, če se slučajno ujemajo z vašo. Pač pa je ta del namenjen tistim, ki letos ponovno oddajate temo, ki vam jo je KŠZ potrdila že lani.
% v kolikor gre za temo, ki je bila že oddana v preteklem letu in je bila takrat potrjena, prosim to napišite. Prav tako napišite, če se v nečem tema razlikuje od lanske (ste kaj dodali, odvzeli).

\subsection{Uvod in opis problema}

%Navodilo:
Pojasnite, kaj je problem, ki ga želite reševati, in podajte motivacijo za delo. Pri opisu motivacije se navežite na literaturo in nerešene probleme, ki jih bo naslavljala vaša magistrska naloga. Delo umestite v ožje področje dela. Besedilo naj obsega približno 800 znakov s presledki vred.

\subsection{Pregled sorodnih del}

%Navodilo:
Opišite pregled sorodnih del na ožjem področju, na katerem nameravate opravljati magistrsko nalogo. Vsako delo naj bo na kratko opisano v nekaj stavkih, besedilo pa naj poudari njegove glavne prednosti, slabosti ali posebnosti. Sklicujte se na dela, navedena v razdelku \ref{literatura} Literatura in viri. Pregled naj bo fokusiran in naj obsega približno pol strani A4.

\subsection{Predvideni prispevki magistrske naloge}

%Navodilo:
Opišite predvidene prispevke magistrske naloge s področja računalništva in informatike, ki so lahko strokovni ali znanstveni. Poudarite in opišite predvideni napredek ali novost vašega dela v primerjavi z obstoječim stanjem na strokovnem (ali znanstvenem) področju. Opis naj obsega približno 500 znakov s presledki vred.


\subsection{Metodologija}

%Navodilo:
Na kratko opredelite metodologijo, ki jo nameravate uporabiti pri svojem delu. Metodologija vsebuje metode, ki jih nameravate uporabiti (npr. razvoj v izbranem programskem jeziku, izdelava strojne opreme itd.), postopek analize, postopek evalvacije vašega prispevka in primerjavo s sorodnimi deli. Opis naj obsega približno 500 znakov s presledki vred.


\subsection{Literatura in viri}
\label{literatura}

%Navodilo:
Tu navedite vse vire, ki jih citirate v predlogu teme. Citiranje naj bo v skladu z znanstveno-strokovnimi standardi citiranja, na primer, \cite{Zivkovic2004}. Seznam naj vsebuje vsaj nekaj del, objavljenih v zadnjih petih letih. Prednostno naj bodo navedene objave s konferenc, revij, oziroma drugih priznanih virov.

\renewcommand\refname{}
\vspace{-50px}
\bibliographystyle{elsarticle-num}
\bibliography{./bibliografija/bibliography}


%\bigskip
%
%Ljubljana, \today .

\end{document}
